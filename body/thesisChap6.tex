\iffalse
\bibliography{reference/refs}
\fi

\chapter{Conclusion and Future Works}
\label{chap:conclusions}

In this thesis, we focus on the privacy protection problem for still images and videos
by de-identifying face regions. To make the de-identification algorithm more robust to
various images, multi-linear algebra is used to store the data. Hence, the algorithm
is more practical. 

For the still imags, we propose a novel algorithm of face de-identification based on 
tensor CP decomposition. The algorithm helps to decompose a face image into multiple 
dimensions so that the de-identification process could happen only on privacy 
related information. All the experiments are conducted in the dataset which 
include multiple images for each person. On the other aspect, the proposed algorithm 
could process the dataset with multiple factors such as {\it pose, expressions}. 
Theoretically, more factors could be appended to the algorithm. Futhermore, the rank-$R$
approximation enlarges the model representation ability for images with new features.
One shortcoming is that the proposed algorithm requires a complete dataset to construct
a tensor. The complete dataset is not easy to collect for some cases. Therefore, the future 
work might be the research on de-identification based on tensors with missing values 
so that the tensor model could break the limitation of incomplete image data. 

For videos, our proposed algorithm use AAM and tensor HOSVD decomposition to de-identify 
a video by replacing the original faces with other natural ones. Shown as the experiments, 
we can keep the identity of de-identified frames in a video invariant and protect its data 
utility, such as expressions simultaneously. There are shortcomings in the proposed method. 
The transformation frames between expressions, such as from neutral to surprise, creates 
unnatural results sometimes. This problem might be solved by apply a weight average process 
to basis selection. This work could be improved by using Tensor-AAM. More data utilities 
could be added to the model.