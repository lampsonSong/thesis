%%%%%%%%%%%%%%%%%%%%%%%%%%%%%%%%%%%%%%%%%%%%%%%%%%%%%%%%%%%%%%%%%%%
%                                                                 %
%                            CHAPTER ONE                          %
%                                                                 %
%%%%%%%%%%%%%%%%%%%%%%%%%%%%%%%%%%%%%%%%%%%%%%%%%%%%%%%%%%%%%%%%%%%

\chapter{Introduction}

The problem that keeps balance between privacy protection and expression preservation is addressed in this paper. 
Face de-identification 

\par
Active appearance model (AAM) have been successfully applied to model the space of human face for decades. The AAM could be used in both face recognition and facial expression recognition methods. What is the difference role in the two recognition methods?(2015-5-11)

\par
The practical situations are eager to find a way to protect human identity privacy. Mosaic and blur.

\par
Imaging a situation in a social experiment, the researchers wish to observe volunteers' direct reactions to some events. As an important channel in communication, facial expression is not tolarable to be ignored for it is the most direct reflection of a person's emotional state. 

\par
Mehrabian illustrates that the spoken words of a message contributes only for $7\%$ to the effect of the message as a whole, the voice intonation contributes for $38\%$, while the facial expression of the speaker contributes for $55\%$ to the effect of the spoken message ~\cite{Meh68}

\par
Hospital surveillance videos are expected to contain rich health information including clinical reactions, effects after some treatment which are anaylzed from behaviors or expressions of the patients. Among the information, the identities of patients are so sensitive that it is forbidden to share the surveillance video with other researchers. For this reason, an technique that could protect patients' identity privacy and preserve most useful expression and behavior information simutanously are eager to be found.