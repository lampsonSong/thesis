\documentclass[a4paper,dvips,12pt,oneside,onecolumn,final,openright]{book}

\title{\textbf{Face De-identification with Preserving Expressions in Different Poses}}
\author{\textbf{Song Yuhao}}
\date{September 2015}


%%----------------------------------------------------------------------------%%
%%    Packages to include                                                     %%
%%----------------------------------------------------------------------------%%


% for general layout
%\usepackage{times}
%\usepackage{thesisStyle}
\usepackage{calc}
\usepackage{tocbibind}
\usepackage{indentfirst}
\usepackage{fancyhdr}
\usepackage{setspace}
\usepackage{titlesec}
\usepackage[font=footnotesize, labelfont={bf}, margin=1cm]{caption}
% for tables
\usepackage{array}
\usepackage{dcolumn}
\usepackage{multirow}
\usepackage{tabularx}
\usepackage{picinpar}
\usepackage{longtable}
% for maths
\usepackage{amsmath}
\usepackage{amssymb}
\usepackage{theorem}
\usepackage{bm}
% for SI symbols
%\usepackage[thinspace,amssymb]{SIunits}

% for graphics
\usepackage{color}
\usepackage{graphicx}
\DeclareGraphicsExtensions{.eps,.pdf,.jpg,.png}
\DeclareGraphicsRule{.png}{eps}{.bb}{}
%\usepackage[tight,FIGBOTCAP,TABBOTCAP,hang]{subfigure}
\usepackage[footnotesize]{subfigure}
\usepackage{rotating}


% for algorithms
%\usepackage[plain,section]{algorithm}
\usepackage{algorithm}
\usepackage{algorithmic}
\usepackage{ifthen}
%%----------------------------------------------------------------------------%%
%%    Page layout                                                             %%
%%----------------------------------------------------------------------------%%
\renewcommand{\cleardoublepage}{\clearpage}
%% from setspace.sty: 1.5 spacing for 11pt font
\renewcommand{\baselinestretch}{1.5}

%% from vmargin.sty
\setlength{\hoffset}{-1in} \setlength{\voffset}{-1in}
\setlength{\footskip}{2.5cm} \setlength{\headsep}{1.5cm}
\setlength{\topmargin}{1cm}
\setlength{\headheight}{2\baselineskip}
\setlength{\oddsidemargin}{3.5cm}
\setlength{\evensidemargin}{3.5cm}
\setlength{\textwidth}{\paperwidth-\oddsidemargin-\evensidemargin}
\setlength{\textheight}{\paperheight-\topmargin-\headheight-\headsep-\footskip-1in}
\setlength{\parskip}{1.5ex} \setlength{\parindent}{2em}
\setlength{\partopsep}{5pt}

% the right margin of TOC, except the line with page number
\makeatletter
\renewcommand{\@tocrmarg}{.5in}
\makeatother

% subfigure and subtable are listed in TOC also
\setcounter{lofdepth}{2}\setcounter{lofdepth}{2}
\setcounter{tocdepth}{3} \setcounter{secnumdepth}{3}


% from subfigure.sty: just to make the label field wider
\makeatletter
\renewcommand{\l@subfigure}{%
  \@dottedxxxline{\ext@subfigure}{2}{3.9em}{3.1em}}
\renewcommand{\l@subtable}{%
  \@dottedxxxline{\ext@subtable}{2}{3.9em}{3.1em}}
\makeatother

% new math operator for CI3/5/7
\DeclareMathOperator*{\sgn}{sgn}


%%----------------------------------------------------------------------------%%
%%    Various environments                                                    %%
%%----------------------------------------------------------------------------%%

\theoremstyle{break} \theorembodyfont{\normalfont}
\newtheorem{definition}{Definition}[section]
\newtheorem{theorem}{Theorem}[section]

\newcommand{\QED}{%
  \ifmmode % if math mode, assume display: omit penalty etc.
  \else \leavevmode\unskip\penalty9999 \hbox{}\nobreak\hfill
  \fi
  \quad\mbox{\rule[0pt]{1.5ex}{1.5ex}}}
\newcommand{\disableQED}{\renewcommand{\QED}{}}
\newenvironment{proof}%
  {\setlength{\parindent}{0pt}\textbf{Proof: }}%
  {\QED}

\newcounter{cases}
\newenvironment{Cases}%
  {\begin{list}%
    {Case \arabic{cases}:}%
    {%
      \usecounter{cases}%
      \setlength{\labelsep}{5pt}%
      \settowidth{\labelwidth}{Case }
      \setlength{\leftmargin}{\labelwidth}
      \advance\labelwidth 20pt
      \advance\leftmargin 25pt
      \setlength{\listparindent}{0pt}%
      \setlength{\itemindent}{0pt}%
    }%
  }%
  {\end{list}}


%%----------------------------------------------------------------------------%%
%%    Environment for Title Page                                              %%
%%----------------------------------------------------------------------------%%

\makeatletter
\renewcommand{\maketitle}
{%
  \begin{titlepage}
    %\renewcommand{\baselinestretch}{1}%
    \begin{center}
      {\LARGE\@title\par}
      \vspace*{\stretch{5}}
      {\Large\@author\par}
      \vspace*{\stretch{5}}
      \textbf{
        A thesis submitted in partial fulfilment of the requirements\\
        for the degree of\\
        Master of Philosophy\\}
      \vspace*{\stretch{5}}
      \begin{table}[!h]
      \centering
      \begin{tabular}{r@{\textbf{: }}l}
        \textbf{Principal Supervisor}& \textbf{Prof. YUEN Pong Chi}\\
        \textbf{Co-Supervisor}& \textbf{Dr. ZHANG Hui}\\
      \end{tabular}
      \end{table}
      \textbf{Hong Kong Baptist University\\}
      {\large\textbf{\@date}\par}
    \end{center}
  \end{titlepage}
} \makeatother

%%----------------------------------------------------------------------------%%
%%    Environment for Declaration                                             %%
%%----------------------------------------------------------------------------%%

\makeatletter
\newcommand{\makedeclaration}
{
\begin{center}\textbf{DECLARATION}\end{center}
\addcontentsline{toc}{chapter}{Declaration}

I hereby declare that this thesis represents my own work which has
been done after registration for the degree of MPhil at Hong Kong
Baptist University, and has not been previously included in a thesis
or dissertation submitted to this or any other institution for a degree,
diploma or other qualifications.
\vspace*{2.5cm}
\begin{flushright}
\begin{tabular}{r@{: }l@{}}
Signature& \underline{\hbox to 30mm{}} \\
Date&\@date\\
\end{tabular}
\end{flushright}
%\pagestyle{plain}%
} \makeatother

%%----------------------------------------------------------------------------%%
%%    Environment for Abstract                                                %%
%%----------------------------------------------------------------------------%%

\makeatletter
\newenvironment{abstract}%
{
\if@twocolumn
    \@restonecoltrue\onecolumn
  \else
    \@restonecolfalse\newpage
  \fi
  \begin{table}
  \begin{center}
  \begin{tabular*}{1\textwidth}{ @{\extracolsep{\fill}} lr }
  SONG Yuhao& Master of Philosophy\\
  Hong Kong Baptist University&\@date\\
  \end{tabular*}
  \end{center}
  \end{table}
  \begin{center}
  \large
  \@title\\
  \vspace*{1.5cm}
  \textbf{Abstract}
  \vspace*{2.5cm}
  \end{center}
%Start Abstract
\addcontentsline{toc}{chapter}{Abstract}
\input{preface/thesisAbs}
}%
\makeatother

%%----------------------------------------------------------------------------%%
%%    Environment for Acknowledgements                                             %%
%%----------------------------------------------------------------------------%%

\makeatletter
\newcommand{\makeAck}
{
\chapter*{Acknowledgements}
\addcontentsline{toc}{chapter}{Acknowledgements}
%%%%%%%%%%%%%%%%%%%%%%%%%%%%%%%%%%%%%%%%%%%%%%%%%%%%%%%%%%%%%%%%%%%%%%%%%%%%%%%
%   Acknowledgements                                                          %
%%%%%%%%%%%%%%%%%%%%%%%%%%%%%%%%%%%%%%%%%%%%%%%%%%%%%%%%%%%%%%%%%%%%%%%%%%%%%%%
%\chapter*{Acknowledgements\markboth{Acknowledgements}{Acknowledgements}}

This project would not have been completed without the support of many people. I highly appreciate my supervisor, Hui Zhang, who enlightened me from knowing nothing in this field and helped me make sense of my confusion. Thanks to my principle supervisor, Pong-Chi Yuen, who offered guidance and support during the period of study. Thanks to Hong Kong Baptist University for providing me a monthly studentship for my study. And finally, thanks to my parents and numerous friends who always offering support and love.



} \makeatother
%%----------------------------------------------------------------------------%%
%%    Environment for Notation                                                %%
%%----------------------------------------------------------------------------%%

\makeatletter
\newcommand{\makenotation}
{
\chapter*{Notation}
\addcontentsline{toc}{chapter}{Notation}
\section*{Camera Parameters}
\begin{table}[!h]
\begin{flushleft}
    \begin{tabular}{ c l }
    $f$ & Focal length\\
    ($u_0,v_0$)& Principal point \\
    \end{tabular}
\end{flushleft}
\end{table}
\section*{Operators}
\begin{table}[!h]
\begin{flushleft}
    \begin{tabular}{ c l }
    $a\times b$ & Cross product of vectors $a$ and $b$\\
    $\textbf{M}^\textbf{T}$ & Transpose of matrix $M$\\
    $\textbf{M}^{-1}$ & Inverse of matrix $M$\\
    $\mp$ & Minus or plus\\
    \end{tabular}
\end{flushleft}
\end{table}
\section*{Named Variables}
\begin{flushleft}
    \begin{longtable}{ c l }
    $\textbf{e}$ & epipole\\
    $\overline{\textbf{a}}$ & Hypothesis of $a$\\
    $\omega$ & Imaged absolute conic\\
    \textbf{K} & Calibration matrix\\
    \textbf{P} & Projection matrix\\
    \textbf{R} & Camera orientation\\
    $\textbf{R}_y$ & Rotation matrix respects to rotation axis\\
    $\textbf{T}$ & Translation vector\\
    $M$ & Mirror\\
    $C$ & Camera\\
    $\Pi$ & The rotation plane\\
    $\textbf{l}_h$ & The horizon line or vanishing line of $\Pi$\\
    $\textbf{m}$ & The point that mirror plane intersects with $\textbf{l}_h$\\
    $\textbf{v}$ & Vanishing point \\
    $\alpha$ & The angle between $C$ and $M_1$\\
    $\beta$ & The angle between $C$ and $M_2$\\
    $\theta$ & The angle between two mirrors\\
    $\textbf{l}_s$ & The imaged rotation axis and the vanishing line of Y-axis\\
    $\textbf{v}_z$ & The vanishing point of Z-axis\\
    $\textbf{v}_x$ & The vanishing point of X-axis\\
    $\textbf{v}_y$ & The vanishing point of Y-axis\\
    \{a,b;c,d\} & Cross ratio of vector $a$, $b$, $c$, $d$\\
    \end{longtable}
\end{flushleft}
%\pagestyle{fancy} \fancyhf{}
%\fancyhead[LO]{\slshape NOTATION}
%\fancyhead[RE]{\slshape NOTATION}
%\fancyfoot[C]{\it \thepage}
} \makeatother
%%----------------------------------------------------------------------------%%
%%    Environment for CV                                                      %%
%%----------------------------------------------------------------------------%%

\makeatletter
\newcommand{\makecv}
{
\if@twocolumn
    \@restonecoltrue\onecolumn
  \else
    \@restonecolfalse\newpage
  \fi
\begin{center}\textbf{CURRICULUM VITAE}\end{center}
\addcontentsline{toc}{chapter}{CURRICULUM VITAE}
Academic qualifications of the thesis author, Mr. CHEN Quan Xin:
\begin{itemize}
  \item Received  the degree of Bachelor of Science(Honours) from Hong Kong Baptist University, November 2011.
\end{itemize}
\vspace*{1cm}
\begin{flushright}
\@date
\end{flushright}
\pagestyle{plain}%
} \makeatother


%%----------------------------------------------------------------------------%%
%%   Various conventions                                                      %%
%%----------------------------------------------------------------------------%%

\newcommand{\NI}{\noindent}
\newcommand{\LL}{\ensuremath{\mathcal{L}}}
\newcommand{\RR}{\ensuremath{\mathcal{R}}}
\newcommand{\CC}{\ensuremath{\mathcal{C}}}
\newcommand{\DD}{\ensuremath{\mathcal{D}}}
\newcommand{\TT}{\ensuremath{\mathcal{T}}}
\newcommand{\EE}{\ensuremath{\mathcal{E}}}
\newcommand{\SSS}{\ensuremath{\mathcal{S}}}
\newcommand{\T}[1]{\ensuremath{\mathit{#1}}}
\newcommand{\N}[1]{\ensuremath{\overline{#1}}}
\newcommand{\Count}[2][]%
  {\ensuremath{\T{count}\ifx#1\undefined\else_{#1}\fi({#2})}}
\newcommand{\Support}[2][]%
  {\ensuremath{\T{supp}\ifx#1\undefined\else_{#1}\fi({#2})}}
\newcommand{\GrowthRate}[3]{\ensuremath{\T{growthRate}\ifx#2\undefined\else_%
  {\ifx#1\undefined\else{#1}\rightarrow\fi{#2}}\fi({#3})}}
\newcommand{\substr}{\sqsubset}
\newcommand{\substreq}{\sqsubseteq}
\newcommand{\prefix}{\prec}
\newcommand{\suffix}{\succ}
\newcommand{\emptystr}{\ensuremath{\varepsilon}}

\newcommand{\FUNCNAME}[1]{\item[{#1}]}
\newcommand{\INPUT}{\item[\textbf{Input:}]}
\newcommand{\INPUTT}{\item[\phantom{\textbf{Input:}}]}
\newcommand{\OUTPUT}{\item[\textbf{Output:}]}
\newcommand{\REMARK}{\item[\textbf{Remark:}]}
\newcommand{\REMARKK}{\item[\phantom{\textbf{Remark:}}]}
%\newcommand{\RETURN}{\textbf{return} }
%\newcommand{\STATEE}{\item[]}
\newcommand{\divider}{\par\mbox{}\hrule\mbox{}\par}

\newcommand{\PreserveBackslash}[1]{\let\temp=\\#1\let\\=\temp}
\let\PBS=\PreserveBackslash
\newcommand{\up}[1]{\raisebox{1.5ex}[0pt][0pt]{#1}}
\newcommand{\md}[1]{\multicolumn{2}{c|}{#1}}
\newcommand{\thickerspace}{\thickspace \thickspace}
\newcommand{\blob}{\rule[.2ex]{1ex}{1ex}}

\hyphenation{ac-know-ledge-ment} \hyphenation{data-bases}
\hyphenation{emer-ging} \hyphenation{sub-string}
\hyphenation{sub-strings} \hyphenation{se-quence}
\hyphenation{se-quences} \hyphenation{classi-fier}
\hyphenation{classi-fiers} \hyphenation{classi-fi-ca-tion}
\hyphenation{infre-quent} \hyphenation{research}
\hyphenation{threshold}


%%----------------------------------------------------------------------------%%
%%    Other stuff                                                             %%
%%----------------------------------------------------------------------------%%

\makeatletter
  \renewcommand{\@subcaption}[2]{%
    \begingroup
      \let\label\@gobble
      \def\protect{\string\string\string}%
      \xdef\@subfigcaptionlist{%
        \@subfigcaptionlist,%
        {\numberline {\@currentlabel}%
      \noexpand{\ignorespaces #2}}}%
    \endgroup
  \@nameuse{@make#1caption}{\@nameuse{@the#1}}}
\makeatother


%%----------------------------------------------------------------------------%%
%%    Content of thesis                                                       %%
%%----------------------------------------------------------------------------%%

%%---------------------%%
\begin{document}
%\setboolean{@twoside}{true}
%%---------------------%%
\maketitle
\frontmatter
\pagestyle{fancy} \fancyhf{} %\fancyhead[LE,RO]{\it \thepage}
%\fancyhead[LO]{\slshape \rightmark}
%\fancyhead[RE]{\slshape \leftmark}
\fancyfoot[C]{\it \thepage}
\renewcommand{\headrulewidth}{0pt}
\renewcommand{\footrulewidth}{0pt}
\fancypagestyle{plain}%
{
  \fancyhf{}
  \fancyfoot[C]{\it \thepage}
  \renewcommand{\headrulewidth}{0pt}
  \renewcommand{\footrulewidth}{0pt}
}

\makedeclaration
\begin{abstract}
%\include{preface/thesisAbs}
\end{abstract}
\makeAck
%%---------------------%%

%%---------------------%%
\tableofcontents \listoftables \listoffigures
\makenotation
%%---------------------%%
\mainmatter
%%---------------------%%
\pagestyle{fancy} \fancyhf{} %\fancyhead[LE,RO]{\thepage}
%\fancyhead[LO]{\slshape \rightmark}
%\fancyhead[RE]{\slshape \leftmark}
\fancyfoot[C]{\thepage}
\renewcommand{\headrulewidth}{0pt}
\renewcommand{\footrulewidth}{0pt}
\fancypagestyle{plain}%
{
  \fancyhf{}
  \fancyfoot[C]{\thepage}
  \renewcommand{\headrulewidth}{0pt}
  \renewcommand{\footrulewidth}{0pt}
}

\iffalse
\bibliography{reference/refs}
\fi
%%%%%%%%%%%%%%%%%%%%%%%%%%%%%%%%%%%%%%%%%%%%%%%%%%%%%%%%%%%%%%%%%%%
%                                                                 %
%                            CHAPTER ONE                          %
%                                                                 %
%%%%%%%%%%%%%%%%%%%%%%%%%%%%%%%%%%%%%%%%%%%%%%%%%%%%%%%%%%%%%%%%%%%

\chapter{Introduction}
\label{chap:intro}

Suppose in a social experiment, the designers expect to observe people's
reactions when strangers ask for a help to them. The experiment results are 
recorded as images and videos. To express the research output to public,
the recorded data is required to be shown. In this case, the privacy of 
participants in images or videos must be well protected. Meanwhile, the
data should still have the ability to describe the expected research output.

We concentrate on protecting privacy by de-identifying face regions in images and
videos. In this chapter, we describe the current research works on face 
de-identification and the challenges in section \ref{sec:bg}, then give out
a summary about our approaches in section \ref{sec:approach}, then list
the contributions in section \ref{sec:contri}, at last demonstrate the thesis
outline in section \ref{sec:outline}. 

\section{Background}
\label{sec:bg}
With the development of camera technologies, the image and video acquisition
is becoming easier. Nowadays, mounts of applications are centered around image
data. Surveillance videos are covering more and more places due to the wide 
deployment of camera devices. On the other aspect, the advances of computing
hardwares and computer vision algorithms make it almost effortless to collect, 
store and analyze massive image and video data. The best face recognition 
algorithms, like FaceNet \cite{facenet15}, DeepFace \cite{deepface14}, can 
achive more than $95\%$ accuracy in $LFW$ and Youtube video datasets. It
means that the current machine recognizer perfoms well and stably regardless
of the complex background, variant illuminations and unknown face poses. A
recent research indicates that it is possible to infer personal informations 
from a single face image \cite{FRandNetwork14}. Therefore, privacy protection
rises as an important problem during the sharing of image and video data. 
With the purpose of protecting privacy, images and videos with people visible 
in the scene are prohibited to be shared in some applications. For example,
in Google Streetview Service which offers high quality street view images,
the faces of people in the scene are blurred in the service website currently.
Another proper example is a surveillance system monitoring patients in nursing 
home \cite{nursing06}. The identity of patients in the surveillance video has 
to be removed before sharing. Since the face is one of the most significant 
biometric features for a person, our research focuses on protecting the privacy 
in images and videos through face de-identification, which aims at removing
identity information from faces.

Different challenges exist in face de-identification for images and videos. For 
images, multiple types of information could be extracted from one face image, such as
identity, expression, skin color, gender, age, etc. Among of them, only identity
information is related to privacy. The other types of information are defined
as data utility of a face image. The key point of face de-identification is to
keep the balance between privacy protection and data utility preservation. For
instance, the expression information must be preserved in the images from a medical 
face database aiming at demonstrate the painful faces \cite{mediDB09}. For videos,
one more point is required in face de-identification. As a set of continuous
images, a video is de-identified frame by frame separately. Thus the identity
of all frames after processing should not change among adjacent frames. 
This would disturb the audiences during the video playback. To summarize,
there are two challenges in face de-identification for images and videos:
\begin{enumerate}
	\item Keeping the balance between privacy protection and data utility 
		  preservation for images and videos.
	\item Keeping the de-identified identity invariant for videos.
\end{enumerate}

To overcome the challenges, plenty of related research works have been released. 
The most common method is obfuscating images such as pixelization or blurring 
\cite{Boyle00,Agrawal09}. Because of the simple implementation, the image 
obfuscation method is suitable not only to images, but also to videos, such 
as TV interviews. However, in this approach, the face regions are unreadable 
to humans after obfuscation. Only replacing the 
face region with another natural face could preserve the non-privacy related 
information while removing privacy information. A formal de-identification
algorithm, $k$-same framework, is then proposed \cite{Newton05,Ralph05,Gross08}. 
The algorithm takes the average of $k$ closest faces as the de-identified result. 
As a consequence, the $k$-same framework is only workable to the person specific 
database, in which each person has just one image. Other formal algorithms, such 
as face swapping, face synthetic from multiple persons, fail to address the 
problem of presreving data utilities \cite{swap08,Mosa14}. Except for the
obfuscation methods, all the formal approaches are not suitable to face
de-identification in videos. Therefore, we try to extent the processing 
algorithms to larger databases and videos. 

\section{Summary of Proposed Approaches}
\label{sec:approach}
We use the active appearance model (AAM) \cite{AAM01,Matthews_04} and higher order 
tensors analysis to de-identify the faces in images and videos. The AAM projects
a face image into a vector space and represents it with a set of coefficients.
The model representation helps avoid ghost faces \cite{Gross08}. Considering
the balance between privacy and data utility, we wish to decompose a face image
into multiple dimensions so that only the privacy related factors are altered. 
Therefore, we build up a higher-order tensor and analyze it. 
Tensor analysis, also known as multilinear algebra, makes the assumption that 
images are formed as the result of multiple factors. Furthermore, these factors 
are amenable to linear analysis as each factor is allowed to vary in turn, while 
the remaining factors are held constant \cite{Vasi02,VasilescuT03}.

For images, the face de-identification algorithm is proposed based on the tensor 
CANDECOMP/PARAFAC(CP) decomposition \cite{Lathauwer_rank,Kolda09}. We construct 
a tensor using multiple types of images, then 
project a new input one into this tensor. After representing the input image by 
identity and other types of parameters, we can pick out the identity factor and 
fuse it with other identity parameters from different persons. At last, we de-identify 
a face image by reconstructing it with altered identity parameters and its 
untouched parameters. The advantage of our tensor-based algorithm is that the 
de-identification process could focus only on privacy related information so that 
the other $data$ $utilities$ could be well preserved. 

Among kinds of tensor decomposition algorithms, the CP decomposition estimates
the parameters for each dimension using Alternating Least Square (ALS) method.
In this way, any face images could be decomposed properly. However, it is not
suitable to face de-identification in videos. The reason is that the CP
decomposition is sensitive to initial values and would produce different
values due to different initial guesses. For videos, AAM is used to represent
face images and the Higher-order Singular 
Value Decomposition(HOSVD) \cite{Lathauwer00} is used to decompose one frame 
into multiple dimensions. One set of them is privacy related, called the 
$identity$ factors, and the others are non-privacy related factors 
\cite{Feng12,TPAMI09}. With the basis subtensors in HOSVD, this approach 
produces the same computation results for every computation. 
During the video de-identification, one frame in the 
video is picked out and its identity factors are altered to produce a set of 
de-identified identity factors. For all the other frames in the video, each 
of them is reconstructed by only replacing the privacy related factors with 
the de-identified identity ones. Therefore, the identity of de-identified 
result could keep constant.

\section{Contributions}
\label{sec:contri}
In this thesis, we have done some work on face de-identification in images
and videos. Our work contributes on three aspects:
\begin{enumerate}
	\item We use tensor analysis in face de-identification. By decomposing
		the face regions into privacy related factors and other non-privacy related 
		factors, the proposed algorithms could focus on removing privacy 
		information so that all the other factors are leaved untouched. Therefore, our approach
		is suitable to the datasets with multiple factors, such as {\it 
		expressions, poses, illuminations,} etc. Furthermore, each person in
		the dataset could have more than one image.
	\item In image processing, our algorithm is workable to the images not 
		involving in the tensor. The tensor CP decomposition is used to process images. 
		Since the parameters for each dimension is estimated by initial guess using ALS,
		the CP decomposition algorithm enlarges the representation ability for face
		images. We also firstly use	rank-$n$ approximation.
		The increasement of value $n$ helps the face representation more precise, especially
		for the images never appear in tensors. 
	\item We succeed to extend the face de-identification algorithms to videos. The existing
		methods add obfuscations to the regions related to privacy. Compared to the
		priveous algorithms, the proposed one has two advantages. 
		Firstly, our algorithm produces natural de-identified results from videos. 
		Because the non-privacy related factors are untouched during de-identification, 
		the results could keep the data utility such as {\it expressions, skin colors,} etc. 
		in a face. Secondly, our method could de-identify a series of images and keep 
		the resulting identity invariant throughout the video playback.
\end{enumerate}

\section{Thesis Outline}
\label{sec:outline}
In this thesis, we develope a framework to remove privacy related factors and 
preserve other data utility factors in images and videos. 
The rest of this thesis is structured as following.
In Chapter \ref{chap:relatedWorks}, we demonstrate the previous research works
on face de-identification in images and videos individually. Chapter \ref{chap:foundamental}
introduces foundamental theories of the proposed algorithm: tensor
analysis and Active Appearance Model. In Chapter \ref{chap:FDimages} and 
Chapter \ref{chap:FDvideos}, the face de-identification in images and videos are 
explained separately. Chapter \ref{chap:conclusions} expresses the summary of our
works and give out conclusions.


\chapter{Related Work}

\section{Ad-hoc methods}
	Simple distortion methods are applied to the image privacy protection field such as pixelation and blurring. The pixelation method subsamples the image and the blurring method smoothes images with some filters like Gaussian filter or average filter ~\cite{Agrawal09,Boyle00}. Both by destorying the original image information, pixelation and blurring make a trade between privacy information and image quality. Although these two methods are widely used in practical situations like TV interviews, they suffer from the same risk proposed in ~\cite{Newton05} called parrot attack which applies face recognition algorithms to pixelated or blurred images so that target images and database images are preprocessed with the similar procedures. 

	Two more de-identificaiton methods are introduced in ~\cite{Winkler14}: blanking and encryption. 
\section{K-anonymity based methods}

\chapter{Proposed Methodology}

Procrustes analysis, named after a bandit from Greek mythology who streches or cut off his victim to his bed, is used to analyse the distribution of a set of shapes. Shape is all geometrical information that remains when location, scale and rotational effects are filtered out from an object ~\cite{IMM2002-0403}. 

For ASM (Active Shape Model), the training data set contains arbitrary size images for the Procrustes analysis would normalize the data.

\section{Background Knowledge}
AAM (Active Appearance Model)

Diagram of the method
%\chapter{Experiment and Results}

he proposed face de-identification approach has been implemented. Recently, I spend most time in setting up experiments and the corresponding analysis. This report would display the de-identified face images in both shape and appearance level. To inspect the ability of protecting privacy, face recognition algorithm is applied to the original face images and de-identified results. On the other hand, expression recognition algorithm is used to check whether the approaches are able to preserve the useful information.  


\section{De-identified results}
In our experiment, CMU PIE face database is firstly used to test the proposed approaches. Images from 20 subjects are chosen to build the tensor model. Each one of them has 3 poses: frontal, left profile, right profile, and 2 expressions: normal and smile. Other images are for testing. The program is executed in Matlab. The Sandia tensor toolbox is used to analyze the tensor. To show the flexibility of our proposed framework, another face database, IMM, is also tested. The types of pose and expression involved in IMM are the same as CMU PIE, but there are no smile images for right profile and left profile poses. Therefore, expression dimension is not included in IMM experiment. 

A face image is processed in two levels: shape and appearance. Similar with AAM, the feature points in one face compose its shape. Different faces in different poses yield a unique shape. The de-identified algorithms could transform a shape into a different one that represents a similar pose. Appearances are the pixels in an image. The de-identified appearance is expected to be different and normal. In plain words, the result face is normal looking. No distortion or other unnatural appearance would show. The most important is people could not recognize the person identity in results based on original images.


	\subsection{Proposed approach}

	All the group images appear in this report is arranged as: the right is the original image and the left is de-identified result. Fig.\ref{fig:shape_recon} is a sample of shape de-identification. The shape data has been decomposed into multiple dimensions and we only modify the identity one, so the result is not the same with original shape but is in similar pose.

	\begin{figure}[!htb]
		  \centering
		  \includegraphics[width=0.6\textwidth]{figure/shape_recon}
		  \caption{Shape de-identification by proposed approach}
		  \label{fig:shape_recon}
	\end{figure}

	For the face shapes are not in the same size, all appearances are warped to a mean shape so that it is possible to compare different images. Therefore, the de-identified appearance should warp back to de-identified shape. That is the final result of our approach. Fig. \ref{fig:tensor_result_1} is one example of final result.

		\begin{figure}[!htb]
		  \centering
		  \includegraphics[width=0.6\textwidth]{figure/right_smile}
		  \caption{De-identified appearance warp back to de-identified shape}
		  \label{fig:tensor_result_1}
		\end{figure}

	The criteria of judgeing the result are:
	\begin{enumerate}
		\item the de-identified face is not the same one with the original face,
		\item the de-identified face is normal, containing no unnatural appearance like distortion.
	\end{enumerate}


	\subsection{k-same-M}
	K-same-M is one of the best face de-identification algorithms. It describes a face image by parameters of a model like AAM and then take the average of $k$ image parameters as a new one. At last, a new face image is reconstructed by the new parameters and the model. Since all parameters are in the same dimension, the result might lose more useful information than identity. Fig. \ref{fig:shape_k_s} is an example of shape de-identification. The original left profile pose is tranformed into frontal face. 

	\begin{figure}[!htb]
	  \centering
	  \includegraphics[width=0.6\textwidth]{figure/model_shape_1}
	  \caption{shape de-identification by k-same-M}
	  \label{fig:shape_k_s}
	\end{figure}

	De-identified appearance that warps back to de-identified shape forms the final result. If the face shape has been tranformed into a wrong one, the final result must be distorted and unnatural. Fig. \ref{fig:MkS} is an example of model-based de-identification results. The result has an obvious distortion in the face. 

	\begin{figure}[!htb]
	  \centering
	  \includegraphics[width=0.6\textwidth]{figure/model_result}
	  \caption{de-identification by k-same-M}
	  \label{fig:MkS}
	\end{figure}

	K-same-M can definitely produce good results like Fig.\ref{fig:model_result}. However, the data utility of de-identified results can not be guaranteed. 

		\begin{figure}[!htb]
		  \centering
		  \includegraphics[width=0.6\textwidth]{figure/good_MkS}
		  \caption{model k same}
		  \label{fig:model_result}
		\end{figure}

\section{Recognition}

We have compared the proposed approach and existing ones in human vision level. The key point in face de-identification is to keep the ballance between privary protection and data utility preservation. We have addressed the expression is the most importatn information.This section would examine privacy and expression preservation in quantity.
	
	
	\begin{figure}[!htb]
	  \centering
	  \includegraphics[width=0.8\textwidth]{figure/plotPic}
	  \caption{Recognition ratio.}
	  \label{fig:plotPic}
	\end{figure}

	Only frontal face images are used because the recognition algorithms performs best in these images. The PCA+KNN is chosen as the inspection method for face recognition (FR) and expression recognition (ER). This is the most stable recognition framework. Furthermore, it is easy to implement and performs well in small dataset. I think the inpsection method is not the key point of this experiment. It is the comparisons that really matter. 

	From the Fig. \ref{fig:plotPic}, we can see FR ratio of proposed approach is higher than k-same-M, but it is still acceptable. The ER ratio of proposed approach is increased to a high level. It is a trade-off between privacy and data utility. Our target is to preserve the data utility as large as possible when the privacy is protected well. 



The above is the objective testing. Further work about subjective testing is being prepared. I would design some questionnares for a survey about FR of de-identified results. 

%\chapter{Conclusion and Future Works}


%%%---------------------%%
%\appendix
%%%---------------------%%
%\include{thesisApd}
%

%%---------------------%%
%\backmatter
%%---------------------%%
\bibliographystyle{splncs03}
\bibliography{reference/refs}
\makecv
\pagestyle{empty}
%\begin{thebibliography}{999}
%\addcontentsline{toc}{chapter}{\numberline{}\bibname}
%\include{tqwangBib}
%\end{thebibliography}

%%---------------------%%
\end{document}
%%---------------------%%


%%----------------------------------------------------------------------------%%
%%    End of thesis                                                           %%
%%----------------------------------------------------------------------------%%
